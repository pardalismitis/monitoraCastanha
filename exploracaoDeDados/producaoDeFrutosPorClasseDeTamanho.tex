% Options for packages loaded elsewhere
\PassOptionsToPackage{unicode}{hyperref}
\PassOptionsToPackage{hyphens}{url}
%
\documentclass[
]{article}
\usepackage{amsmath,amssymb}
\usepackage{lmodern}
\usepackage{iftex}
\ifPDFTeX
  \usepackage[T1]{fontenc}
  \usepackage[utf8]{inputenc}
  \usepackage{textcomp} % provide euro and other symbols
\else % if luatex or xetex
  \usepackage{unicode-math}
  \defaultfontfeatures{Scale=MatchLowercase}
  \defaultfontfeatures[\rmfamily]{Ligatures=TeX,Scale=1}
\fi
% Use upquote if available, for straight quotes in verbatim environments
\IfFileExists{upquote.sty}{\usepackage{upquote}}{}
\IfFileExists{microtype.sty}{% use microtype if available
  \usepackage[]{microtype}
  \UseMicrotypeSet[protrusion]{basicmath} % disable protrusion for tt fonts
}{}
\makeatletter
\@ifundefined{KOMAClassName}{% if non-KOMA class
  \IfFileExists{parskip.sty}{%
    \usepackage{parskip}
  }{% else
    \setlength{\parindent}{0pt}
    \setlength{\parskip}{6pt plus 2pt minus 1pt}}
}{% if KOMA class
  \KOMAoptions{parskip=half}}
\makeatother
\usepackage{xcolor}
\IfFileExists{xurl.sty}{\usepackage{xurl}}{} % add URL line breaks if available
\IfFileExists{bookmark.sty}{\usepackage{bookmark}}{\usepackage{hyperref}}
\hypersetup{
  pdftitle={Monitoramento Participativo da Biodiversidade},
  pdfauthor={Fernando Lima, D.Sc.},
  hidelinks,
  pdfcreator={LaTeX via pandoc}}
\urlstyle{same} % disable monospaced font for URLs
\usepackage[margin=1in]{geometry}
\usepackage{color}
\usepackage{fancyvrb}
\newcommand{\VerbBar}{|}
\newcommand{\VERB}{\Verb[commandchars=\\\{\}]}
\DefineVerbatimEnvironment{Highlighting}{Verbatim}{commandchars=\\\{\}}
% Add ',fontsize=\small' for more characters per line
\usepackage{framed}
\definecolor{shadecolor}{RGB}{248,248,248}
\newenvironment{Shaded}{\begin{snugshade}}{\end{snugshade}}
\newcommand{\AlertTok}[1]{\textcolor[rgb]{0.94,0.16,0.16}{#1}}
\newcommand{\AnnotationTok}[1]{\textcolor[rgb]{0.56,0.35,0.01}{\textbf{\textit{#1}}}}
\newcommand{\AttributeTok}[1]{\textcolor[rgb]{0.77,0.63,0.00}{#1}}
\newcommand{\BaseNTok}[1]{\textcolor[rgb]{0.00,0.00,0.81}{#1}}
\newcommand{\BuiltInTok}[1]{#1}
\newcommand{\CharTok}[1]{\textcolor[rgb]{0.31,0.60,0.02}{#1}}
\newcommand{\CommentTok}[1]{\textcolor[rgb]{0.56,0.35,0.01}{\textit{#1}}}
\newcommand{\CommentVarTok}[1]{\textcolor[rgb]{0.56,0.35,0.01}{\textbf{\textit{#1}}}}
\newcommand{\ConstantTok}[1]{\textcolor[rgb]{0.00,0.00,0.00}{#1}}
\newcommand{\ControlFlowTok}[1]{\textcolor[rgb]{0.13,0.29,0.53}{\textbf{#1}}}
\newcommand{\DataTypeTok}[1]{\textcolor[rgb]{0.13,0.29,0.53}{#1}}
\newcommand{\DecValTok}[1]{\textcolor[rgb]{0.00,0.00,0.81}{#1}}
\newcommand{\DocumentationTok}[1]{\textcolor[rgb]{0.56,0.35,0.01}{\textbf{\textit{#1}}}}
\newcommand{\ErrorTok}[1]{\textcolor[rgb]{0.64,0.00,0.00}{\textbf{#1}}}
\newcommand{\ExtensionTok}[1]{#1}
\newcommand{\FloatTok}[1]{\textcolor[rgb]{0.00,0.00,0.81}{#1}}
\newcommand{\FunctionTok}[1]{\textcolor[rgb]{0.00,0.00,0.00}{#1}}
\newcommand{\ImportTok}[1]{#1}
\newcommand{\InformationTok}[1]{\textcolor[rgb]{0.56,0.35,0.01}{\textbf{\textit{#1}}}}
\newcommand{\KeywordTok}[1]{\textcolor[rgb]{0.13,0.29,0.53}{\textbf{#1}}}
\newcommand{\NormalTok}[1]{#1}
\newcommand{\OperatorTok}[1]{\textcolor[rgb]{0.81,0.36,0.00}{\textbf{#1}}}
\newcommand{\OtherTok}[1]{\textcolor[rgb]{0.56,0.35,0.01}{#1}}
\newcommand{\PreprocessorTok}[1]{\textcolor[rgb]{0.56,0.35,0.01}{\textit{#1}}}
\newcommand{\RegionMarkerTok}[1]{#1}
\newcommand{\SpecialCharTok}[1]{\textcolor[rgb]{0.00,0.00,0.00}{#1}}
\newcommand{\SpecialStringTok}[1]{\textcolor[rgb]{0.31,0.60,0.02}{#1}}
\newcommand{\StringTok}[1]{\textcolor[rgb]{0.31,0.60,0.02}{#1}}
\newcommand{\VariableTok}[1]{\textcolor[rgb]{0.00,0.00,0.00}{#1}}
\newcommand{\VerbatimStringTok}[1]{\textcolor[rgb]{0.31,0.60,0.02}{#1}}
\newcommand{\WarningTok}[1]{\textcolor[rgb]{0.56,0.35,0.01}{\textbf{\textit{#1}}}}
\usepackage{graphicx}
\makeatletter
\def\maxwidth{\ifdim\Gin@nat@width>\linewidth\linewidth\else\Gin@nat@width\fi}
\def\maxheight{\ifdim\Gin@nat@height>\textheight\textheight\else\Gin@nat@height\fi}
\makeatother
% Scale images if necessary, so that they will not overflow the page
% margins by default, and it is still possible to overwrite the defaults
% using explicit options in \includegraphics[width, height, ...]{}
\setkeys{Gin}{width=\maxwidth,height=\maxheight,keepaspectratio}
% Set default figure placement to htbp
\makeatletter
\def\fps@figure{htbp}
\makeatother
\setlength{\emergencystretch}{3em} % prevent overfull lines
\providecommand{\tightlist}{%
  \setlength{\itemsep}{0pt}\setlength{\parskip}{0pt}}
\setcounter{secnumdepth}{-\maxdimen} % remove section numbering
\ifLuaTeX
  \usepackage{selnolig}  % disable illegal ligatures
\fi

\title{Monitoramento Participativo da Biodiversidade}
\author{Fernando Lima, D.Sc.}
\date{}

\begin{document}
\maketitle

\begin{itemize}
\item
  Subprograma Terrestre

  \begin{itemize}
  \item
    Componente Florestal

    \begin{itemize}
    \tightlist
    \item
      Alvo Complementar: Castanha da Amazônia
    \end{itemize}
  \end{itemize}
\end{itemize}

Os dados, scripts e informações aqui contidas estão disponíveis no
GitHub no endereço: \url{https://github.com/pardalismitis/monitora-mpb}

\hypertarget{leitura-de-dados}{%
\subsubsection{Leitura de dados}\label{leitura-de-dados}}

\begin{itemize}
\item
  Produção
\item
  Sementes
\item
  Mapeamento
\end{itemize}

Unir dados de produção e sementes com dados de mapeamento usando o
número das castanheiras como referência.

\texttt{all.x} define que todos os dados de \texttt{producaoCazumba}
serão mantidos.

\begin{Shaded}
\begin{Highlighting}[]
\CommentTok{\#unir com mapeamento para adicionar as classes de tamanho.}
\NormalTok{producaoClasse }\OtherTok{\textless{}{-}} \FunctionTok{merge}\NormalTok{(producaoCazumba,}
\NormalTok{                        mapeamento,}
                        \AttributeTok{by =} \StringTok{"n\_da\_castanheira"}\NormalTok{,}
                        \AttributeTok{all.x =} \ConstantTok{TRUE}
\NormalTok{                        )}
\NormalTok{sementesClasse }\OtherTok{\textless{}{-}} \FunctionTok{merge}\NormalTok{(sementesCazumba,}
\NormalTok{                        mapeamento,}
                        \AttributeTok{by =} \StringTok{"n\_da\_castanheira"}\NormalTok{,}
                        \AttributeTok{all.x =} \ConstantTok{TRUE}
\NormalTok{                        )}
\end{Highlighting}
\end{Shaded}

Converter quilos para gramas.

\begin{Shaded}
\begin{Highlighting}[]
\NormalTok{sementesClasse}\SpecialCharTok{$}\NormalTok{Peso\_total\_das\_sementes\_dos\_10.frutos\_ouricos\_gr }\OtherTok{\textless{}{-}}\NormalTok{ sementesClasse}\SpecialCharTok{$}\NormalTok{Peso\_total\_das\_sementes\_dos\_10.frutos\_ouricos\_Kg}\SpecialCharTok{/}\DecValTok{1000}
\end{Highlighting}
\end{Shaded}

\hypertarget{cuxe1lculo-para-sementes}{%
\subsubsection{Cálculo para sementes:}\label{cuxe1lculo-para-sementes}}

\begin{itemize}
\item
  Número de sementes por fruto
\item
  Média de sementes por fruto
\item
  Peso total e peso de sementes
\item
  Adição de um campo identificador
\end{itemize}

\begin{Shaded}
\begin{Highlighting}[]
\CommentTok{\#soma}
\NormalTok{sementesClasseSoma }\OtherTok{=} \FunctionTok{group\_by}\NormalTok{(sementesClasse, Castanhal, tamanho, Ano)}
\CommentTok{\#cálculos}
\NormalTok{sementesClasseAgregada }\OtherTok{=} \FunctionTok{summarise}\NormalTok{(}
\NormalTok{    sementesClasseSoma,}
    \AttributeTok{total\_sementes =} \FunctionTok{sum}\NormalTok{(}
\NormalTok{        n\_de\_sementes\_por\_fruto\_ourico,}
        \AttributeTok{na.rm =} \ConstantTok{TRUE}
\NormalTok{    ),}
    \AttributeTok{media\_sementes =} \FunctionTok{mean}\NormalTok{(}
\NormalTok{        n\_de\_sementes\_por\_fruto\_ourico,}
        \AttributeTok{na.rm =} \ConstantTok{TRUE}
\NormalTok{    ),}
    \AttributeTok{peso\_total =} \FunctionTok{sum}\NormalTok{(}
\NormalTok{        Peso\_total\_das\_sementes\_dos\_10.frutos\_ouricos\_gr,}
        \AttributeTok{na.rm =} \ConstantTok{TRUE}
\NormalTok{    ),}
    \AttributeTok{peso\_semente =}\NormalTok{  peso\_total}\SpecialCharTok{/}\NormalTok{total\_sementes}
\NormalTok{)}
\end{Highlighting}
\end{Shaded}

\begin{verbatim}
## `summarise()` has grouped output by 'Castanhal', 'tamanho'. You can override
## using the `.groups` argument.
\end{verbatim}

\begin{Shaded}
\begin{Highlighting}[]
\CommentTok{\#adicionando campo de identificação}
\NormalTok{sementesClasseAgregada}\SpecialCharTok{$}\NormalTok{id }\OtherTok{\textless{}{-}} \FunctionTok{paste}\NormalTok{(sementesClasseAgregada}\SpecialCharTok{$}\NormalTok{Castanhal,}
\NormalTok{                                   sementesClasseAgregada}\SpecialCharTok{$}\NormalTok{tamanho,}
\NormalTok{                                   sementesClasseAgregada}\SpecialCharTok{$}\NormalTok{Ano)}
\end{Highlighting}
\end{Shaded}

\hypertarget{cuxe1lculo-para-produuxe7uxe3o}{%
\subsubsection{Cálculo para
produção:}\label{cuxe1lculo-para-produuxe7uxe3o}}

\begin{itemize}
\item
  Número de ouriços produzidos
\item
  Adição de um campo identificador
\end{itemize}

\begin{Shaded}
\begin{Highlighting}[]
\CommentTok{\#soma}
\NormalTok{producaoClasseSoma }\OtherTok{=} \FunctionTok{group\_by}\NormalTok{(producaoClasse, Castanhal, tamanho, Ano )}
\CommentTok{\#cálculos}
\NormalTok{producaoClasseAgregada }\OtherTok{=} \FunctionTok{summarise}\NormalTok{(}
\NormalTok{    producaoClasseSoma,}
    \AttributeTok{total\_frutos =} \FunctionTok{sum}\NormalTok{(}
\NormalTok{        n\_frutos\_ouricos\_produzidos,}
        \AttributeTok{na.rm =} \ConstantTok{TRUE}
\NormalTok{    )}
\NormalTok{)}
\end{Highlighting}
\end{Shaded}

\begin{verbatim}
## `summarise()` has grouped output by 'Castanhal', 'tamanho'. You can override
## using the `.groups` argument.
\end{verbatim}

\begin{Shaded}
\begin{Highlighting}[]
\CommentTok{\#adicionando campo de identificação}
\NormalTok{producaoClasseAgregada}\SpecialCharTok{$}\NormalTok{id }\OtherTok{\textless{}{-}} \FunctionTok{paste}\NormalTok{(producaoClasseAgregada}\SpecialCharTok{$}\NormalTok{Castanhal,}
\NormalTok{                                   producaoClasseAgregada}\SpecialCharTok{$}\NormalTok{tamanho,}
\NormalTok{                                   producaoClasseAgregada}\SpecialCharTok{$}\NormalTok{Ano)}
\end{Highlighting}
\end{Shaded}

\hypertarget{cuxe1lculo-de-produuxe7uxe3o-total}{%
\subsubsection{Cálculo de produção
total}\label{cuxe1lculo-de-produuxe7uxe3o-total}}

União das tabelas de produção e sementes a partir com campo de
identificação e cálculo de produção total.

\begin{Shaded}
\begin{Highlighting}[]
\CommentTok{\#união}
\NormalTok{producaototal }\OtherTok{=} \FunctionTok{merge}\NormalTok{(producaoClasseAgregada,}
\NormalTok{                      sementesClasseAgregada,}
                      \AttributeTok{by =} \StringTok{"id"}\NormalTok{,}
                      \AttributeTok{all.x =} \ConstantTok{TRUE}\NormalTok{)}
\CommentTok{\#cáculo do total produzido}
\NormalTok{producaototal}\SpecialCharTok{$}\NormalTok{producaoTotal }\OtherTok{=}\NormalTok{ producaototal}\SpecialCharTok{$}\NormalTok{total\_frutos}\SpecialCharTok{*}
\NormalTok{  producaototal}\SpecialCharTok{$}\NormalTok{media\_sementes}\SpecialCharTok{*}
\NormalTok{  producaototal}\SpecialCharTok{$}\NormalTok{peso\_semente}
\end{Highlighting}
\end{Shaded}

\begin{Shaded}
\begin{Highlighting}[]
\NormalTok{producaototal }\OtherTok{=} \FunctionTok{as.data.frame}\NormalTok{(producaototal)}
\FunctionTok{write.csv}\NormalTok{(producaototal, }\FunctionTok{here}\NormalTok{(}\StringTok{"output"}\NormalTok{, }\StringTok{"producaoTotal.csv"}\NormalTok{))}
\end{Highlighting}
\end{Shaded}

\hypertarget{cuxe1lculo-de-produuxe7uxe3o-por-castanhal-e-por-classe-de-tamanho}{%
\subsubsection{Cálculo de produção por castanhal e por classe de
tamanho}\label{cuxe1lculo-de-produuxe7uxe3o-por-castanhal-e-por-classe-de-tamanho}}

\begin{Shaded}
\begin{Highlighting}[]
\CommentTok{\#agrupar dados por Castanhal e Ano}
\NormalTok{producaoCastanhal }\OtherTok{=} \FunctionTok{group\_by}\NormalTok{(producaototal, Castanhal.x, Ano.x )}
\CommentTok{\#cálculo da produção}
\NormalTok{producaoCastanhalAgregada }\OtherTok{=} \FunctionTok{summarise}\NormalTok{(}
\NormalTok{    producaoCastanhal,}
    \AttributeTok{total =} \FunctionTok{sum}\NormalTok{(}
\NormalTok{        producaoTotal,}
        \AttributeTok{na.rm =} \ConstantTok{TRUE}
\NormalTok{    )}
\NormalTok{)}
\end{Highlighting}
\end{Shaded}

\begin{verbatim}
## `summarise()` has grouped output by 'Castanhal.x'. You can override using the
## `.groups` argument.
\end{verbatim}

\begin{Shaded}
\begin{Highlighting}[]
\CommentTok{\#agrupar dados por Classe de Tamanho e Castanhal}
\NormalTok{producaoTamanho }\OtherTok{=} \FunctionTok{group\_by}\NormalTok{(producaototal, tamanho.x, Castanhal.x )}
\CommentTok{\#cálculo da produção}
\NormalTok{producaoTamanhoAgregada }\OtherTok{=} \FunctionTok{summarise}\NormalTok{(}
\NormalTok{    producaoTamanho,}
    \AttributeTok{total =} \FunctionTok{sum}\NormalTok{(}
\NormalTok{        producaoTotal,}
        \AttributeTok{na.rm =} \ConstantTok{TRUE}
\NormalTok{    )}
\NormalTok{)}
\end{Highlighting}
\end{Shaded}

\begin{verbatim}
## `summarise()` has grouped output by 'tamanho.x'. You can override using the
## `.groups` argument.
\end{verbatim}

\hypertarget{exportar-arquivos}{%
\paragraph{Exportar arquivos}\label{exportar-arquivos}}

\begin{itemize}
\tightlist
\item
  Produção por castanhal
\end{itemize}

\begin{Shaded}
\begin{Highlighting}[]
\CommentTok{\#producaoCastanhalAgregada = as.data.frame(producaoCastanhalAgregada)}
\FunctionTok{write.csv}\NormalTok{(producaoCastanhalAgregada, }\FunctionTok{here}\NormalTok{(}\StringTok{"output"}\NormalTok{, }\StringTok{"producaoCastanhalAgregada.csv"}\NormalTok{))}
\end{Highlighting}
\end{Shaded}

\begin{itemize}
\tightlist
\item
  Produção por classe de tamanho
\end{itemize}

\begin{Shaded}
\begin{Highlighting}[]
\CommentTok{\#producaoCastanhalAgregada = as.data.frame(producaoCastanhalAgregada)}
\FunctionTok{write.csv}\NormalTok{(producaoTamanhoAgregada, }\FunctionTok{here}\NormalTok{(}\StringTok{"output"}\NormalTok{, }\StringTok{"producaoTamanhoAgregada.csv"}\NormalTok{))}
\end{Highlighting}
\end{Shaded}

\hypertarget{gruxe1fico-de-produuxe7uxe3o-x-ano-x-castanhal}{%
\subsubsection{Gráfico de Produção x Ano x
Castanhal}\label{gruxe1fico-de-produuxe7uxe3o-x-ano-x-castanhal}}

\begin{Shaded}
\begin{Highlighting}[]
\NormalTok{producaoClasseAgregada}\SpecialCharTok{$}\NormalTok{id }\OtherTok{\textless{}{-}} \FunctionTok{paste}\NormalTok{(producaoClasseAgregada}\SpecialCharTok{$}\NormalTok{Castanhal,}
\NormalTok{                                   producaoClasseAgregada}\SpecialCharTok{$}\NormalTok{tamanho,}
\NormalTok{                                   producaoClasseAgregada}\SpecialCharTok{$}\NormalTok{Ano)}
\NormalTok{producaototal }\OtherTok{=} \FunctionTok{as.data.frame}\NormalTok{(producaototal)}
\FunctionTok{write.csv}\NormalTok{(producaototal, }\FunctionTok{here}\NormalTok{(}\StringTok{"output"}\NormalTok{, }\StringTok{"producaoTotal.csv"}\NormalTok{))}
\end{Highlighting}
\end{Shaded}

\begin{Shaded}
\begin{Highlighting}[]
\FunctionTok{ggplot}\NormalTok{(producaoCastanhalAgregada, }\FunctionTok{aes}\NormalTok{(}\AttributeTok{x=}\FunctionTok{factor}\NormalTok{(Ano.x), }\AttributeTok{y=}\NormalTok{total))}\SpecialCharTok{+}
  \FunctionTok{geom\_bar}\NormalTok{(}
    \AttributeTok{position =} \FunctionTok{position\_dodge2}\NormalTok{(}\AttributeTok{preserve =}\StringTok{"single"}\NormalTok{),}
    \AttributeTok{stat=}\StringTok{"identity"}\NormalTok{,}
    \AttributeTok{width =} \FloatTok{0.5}\NormalTok{,}
    \AttributeTok{size =}\FloatTok{0.3}\NormalTok{,}
    \AttributeTok{fill =} \StringTok{"\#006633"}
\NormalTok{    )}\SpecialCharTok{+}
  \FunctionTok{geom\_text}\NormalTok{(}
    \FunctionTok{aes}\NormalTok{(}\AttributeTok{label=} \FunctionTok{as.integer}\NormalTok{(}\FunctionTok{round}\NormalTok{(total, }\DecValTok{0}\NormalTok{))),}
    \AttributeTok{size =} \DecValTok{3}\NormalTok{,}
    \AttributeTok{vjust =} \SpecialCharTok{{-}}\FloatTok{0.5}
\NormalTok{    )}\SpecialCharTok{+}
  \FunctionTok{ylim}\NormalTok{(}\DecValTok{0}\NormalTok{,}\DecValTok{1750}\NormalTok{)}\SpecialCharTok{+}
  \FunctionTok{theme\_grey}\NormalTok{()}\SpecialCharTok{+}
  \FunctionTok{facet\_wrap}\NormalTok{(}\AttributeTok{facets =} \FunctionTok{vars}\NormalTok{(Castanhal.x))}\SpecialCharTok{+}
  \FunctionTok{xlab}\NormalTok{(}\StringTok{"ANO"}\NormalTok{)}\SpecialCharTok{+}
  \FunctionTok{ylab}\NormalTok{(}\StringTok{"PRODUÇÃO (Kg)"}\NormalTok{)}\SpecialCharTok{+}
  \FunctionTok{theme}\NormalTok{(}\AttributeTok{legend.position=}\StringTok{"top"}\NormalTok{)}\SpecialCharTok{+}
  \FunctionTok{theme}\NormalTok{(}\AttributeTok{axis.text.x =} \FunctionTok{element\_text}\NormalTok{(}
    \AttributeTok{angle=}\DecValTok{45}\NormalTok{,}
    \AttributeTok{vjust=}\DecValTok{1}\NormalTok{,}
    \AttributeTok{hjust =} \DecValTok{1}\NormalTok{,}
    \AttributeTok{colour=}\StringTok{"black"}\NormalTok{,}
    \AttributeTok{size=}\FunctionTok{rel}\NormalTok{(}\DecValTok{1}\NormalTok{))}
\NormalTok{    )}
\end{Highlighting}
\end{Shaded}

\includegraphics{producaoDeFrutosPorClasseDeTamanho_files/figure-latex/unnamed-chunk-11-1.pdf}

\hypertarget{gruxe1fico-de-produuxe7uxe3o-x-classe-de-tamanho-x-castanhal}{%
\subsubsection{Gráfico de Produção x Classe de Tamanho x
Castanhal}\label{gruxe1fico-de-produuxe7uxe3o-x-classe-de-tamanho-x-castanhal}}

\begin{Shaded}
\begin{Highlighting}[]
\CommentTok{\#ordenar as classes de tamanho em ordem crescente}
\NormalTok{producaoTamanhoAgregada}\SpecialCharTok{$}\NormalTok{tamanho.x }\OtherTok{\textless{}{-}} \FunctionTok{factor}\NormalTok{(producaoTamanhoAgregada}\SpecialCharTok{$}\NormalTok{tamanho.x,}\AttributeTok{levels =} \FunctionTok{c}\NormalTok{(}\StringTok{"jovem"}\NormalTok{, }\StringTok{"jovem{-}adulto"}\NormalTok{, }\StringTok{"adulto"}\NormalTok{, }\StringTok{"adulto{-}senescente"}\NormalTok{, }\StringTok{"senescente"}\NormalTok{))}
\CommentTok{\#gerar gráfico excluindo NA\textquotesingle{}s}
\FunctionTok{ggplot}\NormalTok{(}\FunctionTok{subset}\NormalTok{(producaoTamanhoAgregada, }\SpecialCharTok{!}\FunctionTok{is.na}\NormalTok{(tamanho.x)), }\FunctionTok{aes}\NormalTok{(}\AttributeTok{x=}\FunctionTok{factor}\NormalTok{(tamanho.x), }\AttributeTok{y=}\NormalTok{total))}\SpecialCharTok{+}
  \FunctionTok{geom\_bar}\NormalTok{(}
    \AttributeTok{position =} \FunctionTok{position\_dodge2}\NormalTok{(}\AttributeTok{preserve =}\StringTok{"single"}\NormalTok{),}
    \AttributeTok{stat=}\StringTok{"identity"}\NormalTok{,}
    \AttributeTok{width =} \FloatTok{0.5}\NormalTok{,}
    \AttributeTok{size =}\FloatTok{0.3}\NormalTok{,}
    \AttributeTok{fill =} \StringTok{"\#006633"}
\NormalTok{    )}\SpecialCharTok{+}
  \FunctionTok{geom\_text}\NormalTok{(}
    \FunctionTok{aes}\NormalTok{(}\AttributeTok{label=} \FunctionTok{as.integer}\NormalTok{(}\FunctionTok{round}\NormalTok{(total, }\DecValTok{0}\NormalTok{))),}
    \AttributeTok{size =} \DecValTok{3}\NormalTok{,}
    \AttributeTok{vjust =} \SpecialCharTok{{-}}\FloatTok{0.5}
\NormalTok{    )}\SpecialCharTok{+}
  \FunctionTok{ylim}\NormalTok{(}\DecValTok{0}\NormalTok{,}\DecValTok{2000}\NormalTok{)}\SpecialCharTok{+}
  \FunctionTok{theme\_grey}\NormalTok{()}\SpecialCharTok{+}
  \FunctionTok{facet\_wrap}\NormalTok{(}\AttributeTok{facets =} \FunctionTok{vars}\NormalTok{(Castanhal.x))}\SpecialCharTok{+}
  \FunctionTok{xlab}\NormalTok{(}\StringTok{"CLASSE DE TAMANHO"}\NormalTok{)}\SpecialCharTok{+}
  \FunctionTok{ylab}\NormalTok{(}\StringTok{"PRODUÇÃO (Kg)"}\NormalTok{)}\SpecialCharTok{+}
  \FunctionTok{theme}\NormalTok{(}\AttributeTok{legend.position=}\StringTok{"top"}\NormalTok{)}\SpecialCharTok{+}
  \FunctionTok{theme}\NormalTok{(}\AttributeTok{axis.text.x =} \FunctionTok{element\_text}\NormalTok{(}
    \AttributeTok{angle=}\DecValTok{45}\NormalTok{,}
    \AttributeTok{vjust=}\DecValTok{1}\NormalTok{,}
    \AttributeTok{hjust =} \DecValTok{1}\NormalTok{,}
    \AttributeTok{colour=}\StringTok{"black"}\NormalTok{,}
    \AttributeTok{size=}\FunctionTok{rel}\NormalTok{(}\DecValTok{1}\NormalTok{))}
\NormalTok{    )}
\end{Highlighting}
\end{Shaded}

\begin{verbatim}
## Warning: Removed 1 rows containing missing values (geom_bar).
\end{verbatim}

\begin{verbatim}
## Warning: Removed 1 rows containing missing values (geom_text).
\end{verbatim}

\includegraphics{producaoDeFrutosPorClasseDeTamanho_files/figure-latex/unnamed-chunk-12-1.pdf}

\hypertarget{gruxe1fico-nuxfamero-de-castanheiras-x-classe-de-tamanho}{%
\subsubsection{Gráfico Número de Castanheiras x Classe de
tamanho}\label{gruxe1fico-nuxfamero-de-castanheiras-x-classe-de-tamanho}}

Selecionando um conjunto de dados (Ano de 2019) como referência para não
somar as classes de todos os anos.

\begin{Shaded}
\begin{Highlighting}[]
\CommentTok{\#selecionar por ano}
\NormalTok{producao2019 }\OtherTok{\textless{}{-}} \FunctionTok{subset}\NormalTok{(producaoClasse, Ano }\SpecialCharTok{==} \DecValTok{2019}\NormalTok{)}
\CommentTok{\#agrupar dados}
\NormalTok{producaoCastanheira }\OtherTok{\textless{}{-}}\NormalTok{ producao2019 }\SpecialCharTok{\%\textgreater{}\%} \FunctionTok{group\_by}\NormalTok{(tamanho, Castanhal) }\SpecialCharTok{\%\textgreater{}\%}\FunctionTok{count}\NormalTok{()}
\end{Highlighting}
\end{Shaded}

\hypertarget{exportar-arquivo}{%
\paragraph{Exportar arquivo}\label{exportar-arquivo}}

\begin{itemize}
\tightlist
\item
  Número de castanheiras x Classes de tamanho
\end{itemize}

\begin{Shaded}
\begin{Highlighting}[]
\FunctionTok{write.csv}\NormalTok{(producaoCastanheira, }\FunctionTok{here}\NormalTok{(}\StringTok{"output"}\NormalTok{, }\StringTok{"producaoCastanheira.csv"}\NormalTok{))}
\end{Highlighting}
\end{Shaded}

\begin{Shaded}
\begin{Highlighting}[]
\CommentTok{\#ordenar as classes de tamanho em ordem crescente}
\NormalTok{producaoCastanheira}\SpecialCharTok{$}\NormalTok{tamanho }\OtherTok{\textless{}{-}} \FunctionTok{factor}\NormalTok{(producaoCastanheira}\SpecialCharTok{$}\NormalTok{tamanho,}\AttributeTok{levels =} \FunctionTok{c}\NormalTok{(}\StringTok{"jovem"}\NormalTok{, }\StringTok{"jovem{-}adulto"}\NormalTok{, }\StringTok{"adulto"}\NormalTok{, }\StringTok{"adulto{-}senescente"}\NormalTok{, }\StringTok{"senescente"}\NormalTok{))}
\CommentTok{\#gerar gráfico}
\FunctionTok{ggplot}\NormalTok{(}\FunctionTok{subset}\NormalTok{(producaoCastanheira, }\SpecialCharTok{!}\FunctionTok{is.na}\NormalTok{(tamanho)), }\FunctionTok{aes}\NormalTok{(}\AttributeTok{x=}\NormalTok{tamanho, }\AttributeTok{y=}\NormalTok{n))}\SpecialCharTok{+}
  \FunctionTok{geom\_bar}\NormalTok{(}
    \AttributeTok{position =} \FunctionTok{position\_dodge2}\NormalTok{(}\AttributeTok{preserve =}\StringTok{"single"}\NormalTok{),}
    \AttributeTok{stat=}\StringTok{"identity"}\NormalTok{,}
    \AttributeTok{width =} \FloatTok{0.5}\NormalTok{,}
    \AttributeTok{size =}\FloatTok{0.3}\NormalTok{,}
    \AttributeTok{fill =} \StringTok{"\#006633"}
\NormalTok{    )}\SpecialCharTok{+}
  \FunctionTok{geom\_text}\NormalTok{(}
    \FunctionTok{aes}\NormalTok{(}\AttributeTok{label=} \FunctionTok{as.integer}\NormalTok{(}\FunctionTok{round}\NormalTok{(n, }\DecValTok{0}\NormalTok{))),}
    \AttributeTok{size =} \DecValTok{3}\NormalTok{,}
    \AttributeTok{vjust =} \SpecialCharTok{{-}}\FloatTok{0.5}
\NormalTok{    )}\SpecialCharTok{+}
  \FunctionTok{ylim}\NormalTok{(}\DecValTok{0}\NormalTok{,}\DecValTok{25}\NormalTok{)}\SpecialCharTok{+}
  \FunctionTok{theme\_grey}\NormalTok{()}\SpecialCharTok{+}
  \FunctionTok{facet\_wrap}\NormalTok{(}\AttributeTok{facets =} \FunctionTok{vars}\NormalTok{(Castanhal))}\SpecialCharTok{+}
  \FunctionTok{xlab}\NormalTok{(}\StringTok{"CLASSE DE TAMANHO"}\NormalTok{)}\SpecialCharTok{+}
  \FunctionTok{ylab}\NormalTok{(}\StringTok{"NÚMERO DE CASTANHEIRAS"}\NormalTok{)}\SpecialCharTok{+}
  \FunctionTok{theme}\NormalTok{(}\AttributeTok{legend.position=}\StringTok{"top"}\NormalTok{)}\SpecialCharTok{+}
  \FunctionTok{theme}\NormalTok{(}\AttributeTok{axis.text.x =} \FunctionTok{element\_text}\NormalTok{(}
    \AttributeTok{angle=}\DecValTok{45}\NormalTok{,}
    \AttributeTok{vjust=}\DecValTok{1}\NormalTok{,}
    \AttributeTok{hjust =} \DecValTok{1}\NormalTok{,}
    \AttributeTok{colour=}\StringTok{"black"}\NormalTok{,}
    \AttributeTok{size=}\FunctionTok{rel}\NormalTok{(}\DecValTok{1}\NormalTok{))}
\NormalTok{    )}
\end{Highlighting}
\end{Shaded}

\includegraphics{producaoDeFrutosPorClasseDeTamanho_files/figure-latex/unnamed-chunk-15-1.pdf}

\end{document}
